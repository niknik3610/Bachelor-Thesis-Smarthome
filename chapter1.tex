\chapter{Introduction} \label{cha:intro}

\section{First Section} \label{sec:intro:first}

First section of your introduction. We would be surprised if there is
no citation to some literature, e.g.,
\cite{lulcs,PapacchiniCaminatiHu23}.

\section{Aims \& Objectives} \label{sec:intro:aims}
When researching available smart home technology, one major gap I came across 
was the availability of open source software. While options exist for someone 
interested in connecting their proprietary device to an open source platform 
(view \todo{find the link to this}), there was no solution for anyone looking to 
build their own device and then connect it to an open source hub. In fulfilling 
this goal, to build an open source platform for both devices and the hub they 
will connect to, there are multiple objectives that will need to be met along 
the way:
\begin{enumerate}
    \item Create a Library and API (Application Programming Interface) for 
        building smart home devices.
    \item Build a Server with an API for the smart home devices to communicate 
        with. This will act as a hub and will control clients connected to it.
         \begin{enumerate}
             \item This API should be well documented, so a user can interact 
                 with the hub, without using the Library.
         \end{enumerate}
     \item Create a frontend, which will be populated with devices currently 
         connected to the smart home. It will also be used to control clients 
         connected to the server.
         \begin{enumerate}
             \item The API provided by the server for this frontend should also 
                 be easy to use, so the user can create their own frontend 
                 environment.
         \end{enumerate}
     \item The code of all of the above should be hosted in a public repository, 
         with instructions for how to build and use every component of the 
         system.
         \begin{enumerate}
             \item An appropriate license should also be selected for this 
                 repository, so the code within it can be copied or modified by 
                 third parties.
             \item This repository should provide important links and provide 
                 information on the inner workings of the system, to support 
                 interested parties.
         \end{enumerate}
\end{enumerate}
    

\begin{lstlisting}[language=Rust, style=colouredRust, caption=Rust Example]
pub fn main() {
    println!("hello world");
}
\end{lstlisting}

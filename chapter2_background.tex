\chapter{Background} \label{cha:chapter2}

\section{IoT System Architectures} \label{sec:chap2:architectures}
Kamienski et al. describe a simple three layer architecture of an IoT system in \cite{DesigningOpenIotSystem}. Within this architecture, the top layer is the "Input System", from which any data that will influence the decisions of the IoT system will come from. Included in this are sensors, but also user facing interfaces. The second layer, known as the "Process System", is where any algorithms are run and system behavioral decisions are made. The goal of this layer is to gain an "improved understanding of the system where the  data comes from" \cite{DesigningOpenIotSystem}. The bottom layer is the "Output System", which are where decisions made by the Process System will be enacted. This is often represented as the devices connected to the IoT system.

This three layer architecture is expanded upon by Bansal and Kumar within \cite{IotEcosystemSurvey}, where three more architectures are described which expand upon the ideas within the three layer architecture. They are however more specialized than the three layer architecture. The first of these is a "Middleware Based" architecture, which can take many forms, but is usually combined with another type of architecture, with a middleware layer. The different types are described in detail by Zhang et al. in \cite{MiddlewareIOTSurvey}. The second is known as a "Fog Based" architecture, where certain tasks, usually those with less processing requirements, are calculated on device to reduce latency. More computationally expensive tasks are however calculated on a server in the cloud \cite{IoTArchitectures}.

The most relevant architecture to this report is known as a "Service Based" architecture (SBA). The SBA is defined around the concept of the Service Oriented Architectural (SOA) style \cite{InteractingSoaBasedIot} of software design. SOA is defined by the Open Group Foundation as an "architectural style that supports service-orientation", where a service is a "logical representation of a repeatable business activity that has a specified outcome" \cite{SoaSourceBook}. Each service is a "black box" any device interacting with it. Other devices use interfaces and API endpoints to make requests to the service and receive a result. A SOA is comprised of many different services. In SBA, services are used to offer device functionality using interfaces, often using web based concepts such as SOAP or REST APIs \cite{TrustManagementSoaIot}. This allows devices with different capabilities and purposes to interact with the same system, allowing for an IoT system that is more flexible. 


\section{The Smart Home System} \label{sec:chap2:smarthome}
Sethi and Sarangi define six components that need to be present within a social IoT setting. A social IoT system is defined as a IoT system where devices form relationships with other devices. While our smart home system will not be a social IoT system, some of these concepts are still of interest. These are: \todo{ask if this is ok with citation as its quite similar}
\begin{enumerate}
    \item ID: the device within the system needs to have a way of identifying 
        it.
    \item Meta-data: the device should have information regarding its form and 
        purpose
    \item Security Controls: the system should have some way of distinguishing 
        between different users. It should also be able to distinguish what 
        types of devices it can connect to or can connect to it.
    \item Service Discovery: each device should be able to discover other 
        devices connected to the system and what services they offer.
\end{enumerate}

There are some specific constraints specific to Smart Homes. Reliability is a key concern, due to the lack of a trained professional being available to fix any issues that arise. This is contrast to more industrial IoT settings, where there might be someone to fix any issues that arise. Another concern is the security and privacy of the system. Due to smart homes inherently having access to sensitive data (due to their position in someone's home), one must ensure that the system is both ethically sound and secure. The issue of security is further discussed in Subsection~\ref{sec:chap2:security}.
\todo{further add to this section, just not sure what yet}

\section{APIs and Web Interfaces} \label{sec:chap2:frontend}
\section{Security} \label{sec:chap2:security}
\section{Networking} \label{sec:chap2:networking}
\section{Open Source and Licensing} \label{sec:chap2:opensource}


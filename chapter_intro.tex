\chapter{Introduction} \label{cha:intro}
The advent of the Internet of Things (IoT) has revolutionized the way we interact with our environment. Devices that were constrained on interaction, be it for practical or aesthetic reasons, can now communicate with not only humans, but also devices surrounding them. Instead of giving devices displays, they can instead communicate through the web, possibly enhancing their utility and making them cheaper to produce. The idea of the IoT emerged in the early 90s from Mark Weiser \cite{FromInternetToIot} and has since paved the way for devices integrated into our everyday lives, in so-called "Smart Homes".

Enabled by the IoT, smart homes are internet connected homes, that allow the user to interact with them in unconventional ways. A lightbulb that is turned on with the users phone instead of a switch, a stereo system that plays music with the press of a button on the users phone. While both these technologies might even seem common place in 2024, they are both directly enabled by IoT and are relatively recent inventions, enabled by the proliferation of wireless technology. While smart home technology may seem mundane at this point, a major issue in the space is the lack of open source standards and frameworks, that not only allow the user to build and connect their own IoT devices to a smart home network, but also provide a server for the devices to connect to and a frontend that allows the user to control the device.

This bachelor thesis aims to delve into the creation of an Open Source smart home system, including a library/framework for the creation of smart home devices, the server for these devices to connect and communicate with, and a web based frontend to control these devices from. It will contain an explanation of various design decisions made throughout this process, exerts and explanations of code from the open source library and test results of the final product. It will also include the code for an example device created to interface with this system, using the device creation framework.



\section{Aims \& Objectives} \label{sec:intro:aims}
When researching available smart home technology, one major gap I came across 
was the availability of open source software. While options exist for someone 
interested in connecting their proprietary device to an open source platform 
(view \todo{find the link to this}), there was no solution for anyone looking to 
build their own device and then connect it to an open source hub. In fulfilling 
this goal, to build an open source platform for both devices and the hub they 
will connect to, there are multiple objectives that will need to be met along 
the way:
\begin{enumerate}
    \item Create a Library and API (Application Programming Interface) for 
        building smart home devices.
    \item Build a Server with an API for the smart home devices to communicate 
        with. This will act as a hub and will control clients connected to it.
         \begin{enumerate}
             \item This API should be well documented, so a user can interact 
                 with the hub, without using the Library.
         \end{enumerate}
     \item Create a frontend, which will be populated with devices currently 
         connected to the smart home. It will also be used to control clients 
         connected to the server.
         \begin{enumerate}
             \item The API provided by the server for this frontend should also 
                 be easy to use, so the user can create their own frontend 
                 environment.
         \end{enumerate}
     \item The code of all of the above should be hosted in a public repository, 
         with instructions for how to build and use every component of the 
         system.
         \begin{enumerate}
             \item An appropriate license should also be selected for this 
                 repository, so the code within it can be copied or modified by 
                 third parties.
             \item This repository should provide important links and provide 
                 information on the inner workings of the system, to support 
                 interested parties.
         \end{enumerate}
\end{enumerate}
    
\section{Project Overview} \label{sec:intro:overview} \todo{ask about this 
section}
\textit{Each bullet point below would give a small summary of the section}
\begin{enumerate}
    \item \textbf{Background Research} 
    \item \textbf{Design of the System \& Technology Decisions}
    \item \textbf{Implementation}
    \item \textbf{Results}
    \item \textbf{Conclusion and Reflection}
\end{enumerate}
